
%%%%%%%%%%%%%%%%%%%%%%%%%%%%%%%%%%%%%%%%%%%%%%%%%%%
\section{Euler's elastica under contact}\label{sec:elastica}
%
This section delves into the problem of an elastica undergoing contact. We will explore the intricacies of this problem and its formulation as a constrained second order Lagrangian problem with augmented Lagrangian, while also discussing the implementation of the associated Python code.
%%%%%%%%%%%%%%%%%%%%%%%%%%%%%%%%%%%%%%%%%%%%%%%%%%%

\subsection{Euler's elastica}
The elastica problem was one of the earliest uses of the calculus of variations, first posed by Jacob Bernoulli and solved by Leonhard Euler, after being introduced to it by his nephew and pupil Daniel Bernoulli \cite{matsutani2010}. It was precisely Daniel who reduced the problem as that of finding a curve which minimizes the functional
\begin{equation*}
    \int_0^l \kappa(s)^2 ds
\end{equation*}
where $\kappa(s)$ denotes the curvature of a curve, and $s, l \in \mathbb{R}$ are the arc-length parameter and total length of the curve respectively. If $q: [0, l] \to \mathbb{R}^2$ denotes a curve parametrized by arc-length, then $\kappa(s) = q''(s)$, where $' = \frac{d}{d s}$.\\

The requirement of arc-length on its changing rate, i.e. $\Vert q'(s) \Vert = 1$, can be introduced as a constraint in the system. Thus, the problem can be stated as a constrained second order Lagrangian problem with augmented Lagrangian $L: T^{(2)}Q \times \mathbb{R} \rightarrow \mathbb{R}$ \cite{singer2008}.

\begin{equation}\label{eq:L_static_elastica}
    L \left( q, q', q'',\Lambda\right) = \frac{1}{2} EI ||q''||^2 + \Lambda (||q'||^2-1) 
\end{equation}

Here, $\Lambda$ is the Lagrange multiplier enforcing the constraint, and the stiffness of the elastica has been parametrized by $E$, its Young's modulus, and $I$, the second moment of area of the beam's cross-section. The continuous action integral $S$ is defined as:

\begin{equation}
    S = \int_{0}^{l} L \left(q,q',q'',\Lambda \right) \,ds 
\end{equation}

Applying Hamilton’s principle of stationary action $\delta S =0$ with fixed boundary conditions $(q(0),q'(0)) = (q_0,q'_0)$ and $(q(l),q'(l)) = (q_N, q'_N)$, yields the Euler-Lagrange equations.

\begin{equation}\label{eq:second_order_ELeq}
    \frac{d^2}{d s^2}  \frac{\partial L}{\partial q''} - \frac{d}{d s}  \frac{\partial L}{\partial q'} + \frac{\partial L}{\partial q} = 0
\end{equation}

The Legendre transformation map $\mathbb{F} L: T^{(3)} Q \rightarrow T^*TQ$ associated with $L(q,q',q'',\Lambda)$ is given by \cite{colombo2016}
\begin{align}
    \mathbb{F} L (q, q', q'', q^{(3)}) = \left( q, q', 
     p = \frac{\partial L}{\partial q'} - \frac{d}{ds} \frac{\partial L}{\partial q''}, 
    \Tilde{p} = \frac{\partial L}{\partial q''} \right)
\end{align}
where $p$ and $\Tilde{p}$ are the conjugate momenta w.r.t. coordinates $q$ and slopes $q'$ respectively. For the elastica, we have
\begin{equation}
    \begin{split}
        p &= - q^{(3)} + 2 \Lambda q' \\
        \Tilde{p} &= q'' 
    \end{split}
\end{equation}
The former is related to axial and transverse forces, while the latter is related to bending moments.
 
The action can be discretised over the length of the beam with space steps $h$, resulting in a discrete Lagrangian formulation. A natural discrete Lagrangian for an augmented second order Lagrangian theory is of the form $L_d: TQ \times TQ \times \mathbb{R} \times \mathbb{R} \rightarrow \mathbb{R}$ \cite{colombo2016}. In particular, we apply the discretisation proposed in \cite{ferraro2021}
\begin{equation}\label{eq:Ld_static_elastica}
    L_d \left( q_i, q'_i, q_{i+1}, q'_{i+1}, \Lambda_i, \Lambda_{i+1} \right) = \frac{h}{2} \left[ L (q_i,q'_i,(q''_i)^\alpha,\Lambda_i) + L (q_{i+1},q'_{i+1},(q''_{i+1})^\alpha, \Lambda_{i+1}) \right]
\end{equation}

where

\begin{align}
    (q''_i)^{\alpha} =& \frac{\left[ (1-3\alpha) q'_{i+1} - (1+3\alpha) q'_i \right] h + 6 \alpha (q_{i+1} - q_i}{h^2} \\
    (q''_{i+1})^{\alpha} =& \frac{\left[ (1+3\alpha) q'_{i+1} - (1-3\alpha) q'_i \right] h - 6 \alpha (q_{i+1} - q_i)}{h^2}
\end{align}

% TODO: Insert here more information about the discrete momenta.

In this study case, $\alpha=1$ is chosen. The discrete action $S_d$ associated to this Lagrangian is
\begin{equation}
    S_d = \sum_{i=0}^{N-1} L_d \left( q_i, q'_i, q_{i+1}, q'_{i+1}, \Lambda_i, \Lambda_{i+1} \right)
\end{equation}
evaluated along the beam. By applying the discrete Hamilton's principle to $S_d$, one obtains the discrete Euler-Lagrange equations, which are the discrete equilibrium equations of the beam.

\begin{equation} 
\begin{split}
    D_1 L \left( q_i,q'_i,q_{i+1},q'_{i+1},\Lambda_{i},\Lambda_{i+1} \right) + D_3 L \left( q_{i-1},q'_{i-1},q_i,q'_i,\Lambda_{i-1}, \Lambda_{i} \right) &= 0 \\
    D_2 L \left( q_i,q'_i,q_{i+1},q'_{i+1},\Lambda_{i},\Lambda_{i+1} \right) + D_4 L \left( q_{i-1},q'_{i-1},q_i,q'_i,\Lambda_{i-1},\Lambda_{i} \right) &= 0 \\
    D_5 L \left( q_i,q'_i,q_{i+1},q'_{i+1},\Lambda_{i},\Lambda_{i+1} \right) + D_6 L \left( q_{i-1},q'_{i-1},q_i,q'_i,\Lambda_{i-1},\Lambda_{i} \right) &= 0
\end{split}
\end{equation}