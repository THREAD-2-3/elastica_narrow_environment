\subsection{Running the code}

The code is implemented in Python 3.9.12 using Spyder 5.1.5 and it is organised as follow: \begin{itemize}
	\item \textit{elastica\textunderscore Lagrangian\textunderscore 2walls}: main code containing the augmented Lagrangian formulation of the elastica in a straight tube, the boundary contidions $(q_0,q'_0)$ and $(q_N, q'_N)$, the solution of the non-linear system and the study of the contact forces;
	\item \textit{param}: definition of parameters such as bending stiffness of the beam $EI$, number of nodes $n\textunderscore nodes$ for space discretisation, parameters $k$ and $p$ for solving the contact problem, $r1$ and $r2$ for definition of the gap function w.r.t. the upper and down walls respectively;
	\item \textit{rigid\textunderscore wall\textunderscore up} and \textit{rigid\textunderscore wall\textunderscore down}: functions for contact problem description w.r.t the upper and down walls respectively;
	\item \textit{mod\textunderscore QVL}: functions useful to modify the structure of the matrix of $q_i$, $q'_i$ and $\lambda_i$.  
\end{itemize}



